
The CrystGap package provides functions for the computation with
affine crystallographic groups, in particular space groups.  For the
definition of the standard crystallographic notions we refer to the
International Tables \cite{Hah95}, in particular the chapter by 
Wondratschek \cite{Won95}, and to the introductory chapter in 
\cite{BBNWZ78}. Some  material  can also be  found in  the chapters 
"The Crystallographic Groups Library" and "Irreducible Maximal Finite
Integral Matrix Groups". The principal algorithms used in this
package are described in \cite{EGN97}.

The present version for {\GAP}~4 has been considerably reworked from
an earlier version for {\GAP}~3.4.4. CrystGap is implemented in the 
{\GAP}~4 language,  and  runs on any system supporting {\GAP}~4.
However, certain commands may require that other share packages
such as CARAT are installed. CARAT is available only under Unix.

As any other share package, CrystGap is loaded with the command
\beginexample 
     gap> RequirePackage( "cryst" ); 
\endexample

CrystGap has been developed by
\beginitems
Bettina Eick &
Fachbereich 17
Universit{\accent127 a}t Gesamthochschule Kassel
D-34109 Kassel, Germany \hfill\break 
e-mail: `eick@mathematik.uni-kassel.de'

Franz G{\accent127 a}hler &
Institut f{\accent127 u}r Theoretische und Angewandte Physik,\hfil\break
Universit{\accent127 a}t Stuttgart,
D-70550 Stuttgart, Germany \hfill\break
e-mail: `gaehler@itap.physik.uni-stuttgart.de'

Werner Nickel &
Mathematik
Universit{\accent127 a}t Darmstadt \hfill\break
Darmstadt, Germany \hfill\break
e-mail: `nickel@mathematik.uni-darmstadt.de'
\enditems

Please send bug reports,  suggestions and other  comments to any of these
e-mail addresses.

The   first and  third  authors acknowledge  financial  support from  the
Graduiertenkolleg {\it Ana\-lyse und Konstruktion in der Mathematik}. The
second author   was supported  by the  Swiss  Bundesamt  f{\accent127 u}r
Bildung und Wissenschaft  in the framework  of the  HCM programme of  the
European   Community.  This collaboration was in   part  made possible by
financial support from the HCM project {\it Computational Group Theory}.




